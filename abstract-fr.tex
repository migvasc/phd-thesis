
L'informatique en nuage est l'une des épines dorsales de notre société numérique, fournissant des ressources informatiques pour la majorité des services et des applications que nous utilisons quotidiennement. Compte tenu de son importance, nous ne pouvons pas négliger son impact environnemental, généré par la consommation d'énergie des centres de données et le cycle de vie de l'infrastructure.

Les principaux fournisseurs de services cloud ont pris des engagements et commencent à déployer des projets pour intégrer de l'électricité renouvelable dans l'opération de leurs centres de données. Cependant, l'un des principaux défis de l'énergie renouvelable est sa nature intermittente --- sa production variant au fil du temps. D'autres facteurs essentiels doivent également être pris en compte. D'abord, l'infrastructure renouvelable présente également un impact environnemental tout au long de son cycle de vie, et chaque emplacement géographique a une capacité différente de production d'énergie renouvelable. De plus, certaines régions du monde ont déjà la présence de sources d'énergie renouvelables dans leur mix énergétique.

Dans cette thèse, nous étudions comment réduire l'impact environnemental --- en termes d'empreinte carbone -- de l'opération et de le dimensionnement des centres de données cloud. Nous explorons des stratégies ``Carbon-responsive'' ---- des approches qui sont conscientes de leurs impacts environnementaux et prennent des décisions éclairées ---- pour élaborer nos solutions proposées.



La première stratégie principale explorée est le  ``follow-the-renewables'', une approche qui alloue et migre la charge de travail vers les centres de données bénéficiant de la plus grande disponibilité d'énergie renouvelable. Nous avons évalué son impact sur le réseau et la consommation d'énergie, et avons proposé un algorithme de planification qui, en tenant compte de la topologie du réseau et de son utilisation, peut planifier les migrations sans générer de congestion réseau ni de gaspillage d'énergie par rapport à des références qui ne tiennent pas compte du réseau.




La deuxième principale stratégie explorée est le dimensionnement de l'infrastructure renouvelable et informatique pour minimiser l'empreinte carbone des centres de données dans le cloud: la définition de la surface nécessaire pour les panneaux solaires, du nombre d'éoliennes, de la capacité des batteries, et du nombre de serveurs . Nous proposons une formulation de programme linéaire qui prend en compte les caractéristiques spécifiques de chaque emplacement géographique des centres de données en termes de capacité de production d'énergie renouvelable, de besoins en refroidissement et de composition du mix énergétique. De plus, nous avons également pris en compte l'impact environnemental de l'infrastructure renouvelable et de la fabrication des serveurs. Notre solution résout les deux sous-problèmes de la planification de la charge de travail en utilisant le ``follow-the-renewables'' et le dimensionnement de l'infrastructure comme un seul problème, ce qui nous permet d'évaluer s'il faut augmenter la capacité de l'infrastructure renouvelable ou allouer la charge de travail à un autre centre de données.

Finalement, le modèle n'utilise que des variables linéaires, ce qui lui permet d'être résolu de manière optimale en temps polynomial. Le modèle est flexible pour évaluer de nombreux scénarios. Par exemple, nous présentons une analyse de la faisabilité de l'inclusion d'éoliennes dans l'infrastructure renouvelable sur site, comment la planification de la charge de travail peut atténuer les erreurs du processus de dimensionnement causées par l'intermittence, les coûts monétaires de la réduction de l'empreinte carbone, et une approche possible pour décider quand fabriquer de nouveaux serveurs en tenant compte de leur impact environnemental. Les décideurs pourraient utiliser ce modèle pour orienter leurs efforts visant à réduire les émissions de carbone des centres de données cloud.