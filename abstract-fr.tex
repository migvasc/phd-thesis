L'informatique en nuage fournit des ressources de calcul pour la plupart des services et applications que nous utilisons au quotidien. Par conséquent, nous devons prêter attention à l'impact environnemental généré par la consommation d'énergie des centres de données (représentant 1\% de la demande mondiale d'électricité) et le cycle de vie de l'infrastructure.


Les principaux fournisseurs de services cloud intègrent l'électricité renouvelable dans le fonctionnement de leurs centres de données pour réduire leur impact environnemental. Cependant, plusieurs facteurs doivent être pris en compte lors de l'utilisation de l'énergie renouvelable : sa production varie dans le temps, l'infrastructure renouvelable présente un impact environnemental pendant son cycle de vie, chaque emplacement géographique a une capacité différente de production d'énergie renouvelable, et certains endroits dans le monde utilisent déjà des sources d'énergie renouvelable dans leur mix énergétique.


Dans cette thèse, nous étudions comment réduire l'impact environnemental, en termes d'empreinte carbone, de l'opération et du dimensionnement des centres de données cloud. Nous explorons des stratégies ``Carbon-responsive'', des approches qui sont conscientes de leurs impacts environnementaux et prennent des décisions éclairées pour élaborer nos solutions proposées.

La première stratégie principale que nous avons explorée est le suivi des énergies renouvelables, une approche qui alloue et migre la charge de travail vers les centres de données avec la plus grande disponibilité de sources d'énergie à faible teneur en carbone. Nous avons évalué son impact sur le réseau et la consommation d'énergie, et avons proposé un algorithme de planification qui, en tenant compte de la topologie du réseau et de l'utilisation, planifie les migrations sans générer de congestion du réseau ni de gaspillage d'énergie.

La deuxième stratégie principale que nous avons explorée est le dimensionnement de l'infrastructure renouvelable et informatique : la définition de la surface nécessaire pour les panneaux solaires, le nombre d'éoliennes, la capacité des batteries et le nombre de serveurs. Nous proposons une formulation de programme linéaire qui prend compte de l'empreinte carbone de l'infrastructure renouvelable et de la fabrication des serveurs, ainsi que des caractéristiques spécifiques de chaque emplacement géographique en termes de capacité de production d'énergie renouvelable, de besoins de refroidissement et de composition du mix énergétique.


Enfin, le modèle n'utilise que des variables linéaires, ce qui lui permet d'être résolu de manière optimale en temps polynomial. Le modèle est flexible pour évaluer de nombreux scénarios. Par exemple, nous présentons une analyse de la réduction potentielle des émissions de carbone grâce à l'adoption de serveurs de nouvelle génération. Les décideurs pourraient utiliser ce modèle pour orienter leurs efforts visant à réduire les émissions de carbone des centres de données cloud.



\textbf{Mots-clés:} l'informatique en nuage; l'informatique verte; follow-the-renewables; dimensionnement; carbon-aware; carbon-responsive computing. 

