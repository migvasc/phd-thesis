Cloud computing is one of the backbones for our digital society, providing computational resources for the marjority of services and applications we use daily --- as e-mail, social-networks, internet banking, video streaming, and health applications. Given its importance, we cannot neglect its environmental impact, generated from the country-like energy consumption of the data centers and the life cycle of the infrastructure. Major cloud providers made commitments and are starting to deploy projects to integrate low-carbon renewable electricity in the operations of their data centers. However, one of the main chalenges of renewable energy is its intermitent nature, that is, its production variates over time influenced by factors as time of the day, seasons and geographic location. Other important aspects that needs to be considered are that the renewable infrastructure also presents environmental impact during its life cycle, and each geographic location has a different capacity for generating renewable power. Additionaly, some locations in the world already have the presence of renewable sources in their energy-mix.

In this thesis, we study how to reduce the environmental impact --- in terms of carbon footprint --- of operating and sizing cloud data centers. We explore carbon-responsive strategies --- approaches that are aware of their environmental impacts and make informed decisions --- to generate our proposed solutions.

The firt main strategy explored is the follow-the-renewables, an approach that allocates and migrate the workload to the data centers with the highest availability of low-carbon-intensive power sources. We evaluated its impact in both the network and energy consumption, and proposed an scheduling algorithm that by taking into account the network topology and the usage can plan the migrations without generating network congestion and wastage of energy --- since the migration process uses a computational task.

The second main strategy explored is the sizing of the renewable and IT infrastructure, that is, defining the required area of solar panels, number of wind turbines and capacity of batteries (to store excess of energy and use when opportune), and number of servers, required to minimize the carbon footprint of the cloud data centers. We propose Linear Program formulation that considers the specific characteriscs of each geographic location of the data centers in terms of capacity for generating renewable power, cooling needs, and the composition of the energy-mix, since some locations already have presence of renewables. Furthermore, we also considered the environmental impact of the renewable infrastructure and manufacturing the servers. 

At a first moment, we evaluated sizing pv only bat

then we extended to the other scenarios ...

model can be used by decision makers ...


Falar do consumo de energia
impacto ambiental dos dcs

emissao de carbono
tecnicas carbon aware

carbon responsive
sizing e contention

follow the renewables


