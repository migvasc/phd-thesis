
A computação em nuvem fornece recursos computacionais para a maioria dos serviços e aplicações que usamos diariamente. Dada a sua importância, não podemos negligenciar o seu impacto ambiental, gerado pelo consumo de energia dos centros de dados (1\% da demanda mundial de eletricidade) e pelo ciclo de vida da infraestrutura.


Grandes provedores de serviços em nuvem estão integrando eletricidade renovável nas operações de seus centros de dados para minimizar o impacto ambiental. No entanto, os seguintes fatores devem ser considerados quando usamos energia renovável: sua produção varia ao longo do tempo; a infraestrutura renovável também apresenta impacto ambiental ao longo de seu ciclo de vida; cada localização geográfica tem uma capacidade diferente de geração de energia renovável; e algumas regiões do mundo já têm a presença de fontes renováveis em sua matriz energética.


Nesta tese, estudamos como reduzir o impacto ambiental --- em termos de pegada de carbono --- da operação e dimensionamento de centros de dados em nuvem. Exploramos estratégias ``carbon-responsive'' --- abordagens que são cientes de seus impactos ambientais e tomam decisões informadas --- para desenvolver nossas soluções propostas.

A primeira estratégia principal explorada é o ``follow-the-renewables'', uma abordagem que aloca e migra a carga de trabalho para os centros de dados com maior disponibilidade de fontes de energia renovável. Avaliamos seu impacto na rede e consumo de energia e propusemos um algoritmo de escalonamento que, considerando a topologia  e o uso da rede, pode planejar as migrações sem gerar congestionamento de rede e desperdício de energia.

A segunda estratégia principal explorada é o dimensionamento da infraestrutura renovável e de TI necessária para minimizar a pegada de carbono dos centros de dados em nuvem: definir a área necessária para os painéis solares, o número de turbinas eólicas, a capacidade das baterias e o número de servidores. Propomos uma formulação de Programa Linear que leva em consideração o impacto ambiental da infraestrutura renovável e da fabricação dos servidores, assim como s características específicas de cada localização geográfica em termos de capacidade de geração de energia renovável, necessidades de resfriamento e composição da matriz energética. 

Por fim, o modelo utiliza apenas variáveis lineares, o que permite que seja resolvido de forma ótima em tempo polinomial. O modelo é flexível para avaliar diversos cenários. Por exemplo, apresentamos uma análise do potencial de redução de emissões de carbono ao adotar servidores de uma nova geração. Os tomadores de decisão podem usar esse modelo para orientar seus esforços na redução das emissões de carbono dos centros de dados de plataformas de computação em nuvem.




\textbf{Palavras-chave:} computação em nuvem; computação verde; follow-the-renewables; dimensionamento; \textit{carbon-aware}; computação \textit{carbon-responsive}. 