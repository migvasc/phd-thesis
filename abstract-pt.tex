
A computação em nuvem é uma das bases da nossa sociedade digital, fornecendo recursos computacionais para a maioria dos serviços e aplicações que usamos diariamente, desde e-mail, redes sociais, internet-banking, vídeo sob demanda e aplicações de saúde. Dada a sua importância, não podemos negligenciar o seu impacto ambiental, gerado pelo consumo de energia dos centros de dados (que poderia suprir anualmente a demanda de países inteiros) e pelo ciclo de vida da infraestrutura.


Grandes provedores de serviços em nuvem fizeram compromissos e estão começando a implantar projetos para integrar eletricidade renovável com baixa emissão de carbono nas operações de seus centros de dados. No entanto, um dos principais desafios da energia renovável é a sua natureza intermitente --- sua produção varia ao longo do tempo, influenciada por fatores como horário do dia, estações do ano e localização geográfica. Outros fatores essenciais também dever ser considerados. Primeiro, a infraestrutura renovável também apresenta impacto ambiental ao longo de seu ciclo de vida, e cada localização geográfica tem uma capacidade diferente de geração de energia renovável. Além disso, algumas regiões do mundo já têm a presença de fontes renováveis em sua matriz energética.


Nesta tese, estudamos como reduzir o impacto ambiental --- em termos de pegada de carbono --- da operação e dimensionamento de centros de dados em nuvem. Exploramos estratégias ``carbon-responsive'' --- abordagens que são cientes de seus impactos ambientais e tomam decisões informadas --- para desenvolver nossas soluções propostas.

A primeira estratégia principal explorada é o ``follow-the-renewables'', uma abordagem que aloca e migra a carga de trabalho para os centros de dados com maior disponibilidade de fontes de energia renovável. Avaliamos seu impacto na rede quanto e consumo de energia e propusemos um algoritmo de escalonamento que, considerando a topologia  e o uso da rede, pode planejar as migrações sem gerar congestionamento de rede e desperdício de energia em comparação com referências que não consideram a rede.

A segunda estratégia principal explorada é o dimensionamento da infraestrutura renovável e de TI necessária para minimizar a pegada de carbono dos centros de dados em nuvem: definir a área necessária para os painéis solares, o número de turbinas eólicas, a capacidade das baterias (para armazenar energia excedente e usá-la quando oportuno) e o número de servidores. Propomos uma formulação de Programa Linear que leva em consideração as características específicas de cada localização geográfica dos centros de dados em termos de capacidade de geração de energia renovável, necessidades de resfriamento e composição da matriz energética, uma vez que algumas localizações já têm presença de fontes renováveis. Além disso, também consideramos o impacto ambiental da infraestrutura renovável e da fabricação dos servidores. Nossa solução resolve os dois subproblemas de escalonamento de carga de trabalho usando ``follow-the-renewables'' e o dimensionamento da infraestrutura como um único problema, o que nos permite avaliar se devemos aumentar a capacidade da infraestrutura renovável ou alocar a carga de trabalho para outro centro de dados.

Por fim, o modelo utiliza apenas variáveis lineares, o que permite que seja resolvido de forma ótima em tempo polinomial. O modelo é flexível para avaliar diversos cenários. Por exemplo, apresentamos uma análise da viabilidade de incluir turbinas eólicas na infraestrutura renovável, como o agendamento de carga de trabalho pode mitigar os erros do processo de dimensionamento causados pela intermitência, os custos monetários de redução da pegada de carbono e uma possível abordagem para decidir quando fabricar novos servidores considerando seu impacto ambiental. Os tomadores de decisão podem usar esse modelo para orientar seus esforços na redução das emissões de carbono dos centros de dados de plataformas de computação em nuvem.




