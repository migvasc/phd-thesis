Cloud computing is an essential component of our modern digital society, given that it supports the majority of services and applications we use. On the other hand, one cannot neglect the environmental impact it presents, which originates both from the energy consumption of its operation---higher than the electricity demand of entire countries---and the life cycle of its infrastructure.

In this thesis, we studied strategies to reduce the environmental impact---in terms of carbon footprint---of operating and sizing cloud federations with data centers geographically distributed worldwide. More specifically, we explored two carbon-responsive strategies: ``follow-the-renewables'' and the sizing of the renewable infrastructure and IT equipment.

This chapter begins by providing an overview of the main contributions made in this thesis, as detailed in Section~\ref{sec:conclusion_summary}. Then, we discuss in Section~\ref{sec:conclusion_future_research} potential opportunities for future research directions that build upon the insights derived from the work presented in this thesis. Finally, Section~\ref{sec:conclusion_dissemination} presents the various ways in which our work has been disseminated, including a list of scientific publications that have emerged from the research conducted in this thesis.


\section{Summary of contributions }

\label{sec:conclusion_summary}

The first main contribution of this thesis is the analysis of the impact of the follow-the-renewables strategy on both energy consumption and network---presented in Chapter~\ref{chap:smartgreens}. This approach is an interesting solution to exploit the geographic distribution of cloud data centers all over the world to reduce high-carbon-intensive electricity consumption. However, the decision to migrate the workload needs to consider the resources involved in this operation, particularly the network. 

We conducted computational experiments to investigate the impact of the follow-the-renewables strategies. These experiments used real-world data for workload, climate conditions, models validated by the scientific community for energy consumption and network usage, and different adaptations of the follow-the-renewables. 

The results demonstrated that migration planning without taking into account the network---in particular, the network topology and usage---will result in network congestion and waste of energy. As seen in Section~\ref{sec:wasted_resources_smartgreens}, the energy wasted could be used to power the servers of cloud data centers, and it also contributes to increasing the environmental impact of the cloud operation. We proposed an algorithm for migration planning that can reduce non-renewable energy consumption while not impacting the network, compared to works from the literature that schedule the workload migration without considering the network.

The second main contribution of this thesis is the model to both size the renewable and IT infrastructure and operate cloud federations with the objective of reducing the carbon footprint---presented in Chapter~\ref{chap:ccgrid} and Chapter~\ref{chap:ccgrid-extension}. 

The initial model proposed in Chapter~\ref{chap:ccgrid} focused on the operation for the short term with one year of duration and the possibility of manufacturing on-site solar panels and lithium-ion batteries to generate renewable power and store it to deal with the intermittency of green power. 

The first key aspect of the proposed solution is that it solves both subproblems---sizing and operating---as a single problem, allowing us to evaluate decisions such as increasing the on-site renewable infrastructure dimensions of a data center or scheduling the workload to different geographic locations using the follow-the-renewables approach.

Another key aspect of the model is that it considers the specific characteristics of the geographic locations of the DCs in terms of climate conditions---the capacity to generate renewable power, cooling needs, and the local grid electricity supply---that may already incorporate low-carbon-intensive power sources. Lastly, the proposed solution does not neglect that manufacturing the renewable infrastructure also presents an environmental impact. 

Through experiments using real-world data for workload, climate conditions, server specifications, location of data centers, and grid energy mix,  we demonstrated that the hybrid solution, which combines the on-site renewable infrastructure and can use power from the local grid, is better in terms of carbon footprint than only operating the DCs using power from the local electricity grid or exclusively power from the on-site renewable infrastructure.

We then extended our modeling for the long term in Chapter~\ref{chap:ccgrid-extension}---which is mandatory given that data centers have operational lifetimes spanning over a decade. We started by improving the modeling to consider the carbon footprint of the entire life cycle of the renewable infrastructure---from manufacturing to discarding or recycling---which provides more accurate information on the environmental impact and reduces oversizing. 

Furthermore, we also evaluated including wind power as an on-site renewable source. Our experiments demonstrated that despite further reducing the carbon footprint, it might not be adequate for on-site renewable production given its requirements in terms of land area, the fact that it increases the uncertainty of the sizing process, and the higher monetary costs for wind power production. Therefore, the best candidates for the on-site renewable infrastructure are solar panels to generate power and lithium-ion batteries to store power to supply at night or during periods with low irradiance. 

We also showed that workload scheduling can mitigate a part of the impact of the errors in the sizing process---caused by the uncertainty of the climate conditions. 

In terms of costs, we showed that adopting on-site renewable infrastructure is cheaper than operating the DCs exclusively with power from the local electricity grid. Additionally, the solution with solar panels and batteries is the best in both environmental and monetary aspects for the cloud operators. 

Our model also enables us to assess the potential carbon reduction by adopting servers from new generations that may be more power-efficient, while also considering the carbon footprint of manufacturing the servers. The model is flexible enough to compare one generation of servers to another or determine what would be the optimal solution given many generations of servers. In the optimal solution, the servers are used over their expected lifetime. This fact can be used as inspiration for extending real-life server usage, since they are computationally powerful, discarding them results in environmental impact, and waiting to replace the servers increases the chance of having a server generation that is more power-efficient and computationally powerful.

Finally, the model was thoughtfully designed to exclusively employ linear variables. This enables the model to be solved optimally in polynomial time and evaluate long-term scenarios in a reasonable amount of time. This efficiency reduces the energy consumption and associated carbon footprint of the sizing process itself. Decision-makers could adopt this model to guide their efforts to reduce the environmental impact of cloud computing platforms.
\section{Future Research Directions}

\label{sec:conclusion_future_research}

As discussed in Section~\ref{sec:conclusion_smargreens}, to further assess the impact of the follow-the-renewables strategies, we also need to take into account the network usage by the workload---that reduces the available bandwidth for the migration process. Additionally, the model can be analyzed for container virtualization technology. Since containers are typically smaller than virtual machines, their migration time is faster. This can result in more tasks being migrated or allow migration to data centers located farther away.


We saw in Chapter~\ref{chap:smartgreens} that applying follow-the-renewables has significant potential for reducing the carbon emissions of the cloud operation. We didn't include it in our modeling for sizing and operating the globally distributed cloud data centers, given the high latency between the network links that connect these data centers, which would result in a long migration time. This analysis needs to be made to evaluate the potential of migrating the workload, given that the DCs located farther away might have more availability of low-carbon intensive sources. One possibility is to have a threshold for the migration time, or migrate only tasks with low priority as batch tasks.


Regarding our proposed approach for sizing and operating cloud federations, one crucial point for the long term is the degradation of the renewable and IT infrastructure and the failure of IT equipment. For the decision to manufacture new servers, the model can be extended to a fine grain at the level of the IT equipment (for example, replacing only a disc that failed). The monetary cost analysis can also be applied to the decision to manufacture or replace new servers, provided that data on the hardware price is accessible. Another possibility is to analyze the sizing process considering the usage of other types of hardware for computations, particularly graphical processors (GPUs) that are widely used for artificial intelligence applications.

We explored the sizing decision for existing cloud federations. Our proposed solution could also be extended to evaluate the placement of new data centers with the objective of minimizing the carbon footprint. For example, given a set of options for locations, their respective characteristics in terms of potential for generating renewable power, climate conditions, cooling needs, workload demand, and server specifications and their associated carbon footprint, it would be possible to perform a comparative analysis of the carbon footprint of each location. The results could be used to guide the decision on the location of the new data center.

Finally, there are other types of environmental impacts other than carbon emissions. For example, cloud computing platforms present impacts in terms of water usage for cooling, resource extraction for manufacturing the integrated circuits and the data center infrastructure, and improper disposal of the IT equipment, which could result in toxic substances being released into the environment. In the first moment, it is necessary to model these types of environmental impacts to generate metrics that will permit us to measure them. Similar to how there are metrics to assess the quality of service, it would be interesting to have metrics to evaluate the environmental impact of these services and applications. In the second step, it will be possible to propose solutions to reduce the environmental impact. For example, these metrics could be incorporated into multi-objective workload scheduling algorithms or included in the sizing process to minimize them.

\section{Work Dissemination}

\label{sec:conclusion_dissemination}

The following two main publications emerged from the work performed during this thesis---both in the format of full papers:

\begin{itemize}

\item  \textit{\textbf{Miguel Felipe Silva Vasconcelos}, Daniel Cordeiro, Georges Da Costa, Fanny Dufossé, Jean-Marc Nicod, and Veronika Rehn-Sonigo. ``Optimal sizing of a globally distributed low carbon cloud federation''. In 2023 IEEE/ACM 23rd International Symposium on Cluster, Cloud and Internet Computing (CCGrid), Bangalore, India, 2023, pp. 203-215. DOI: 10.1109/CCGrid57682.2023.00028}.
  
\item  \textit{\textbf{Miguel Felipe Silva Vasconcelos}, Daniel Cordeiro and Fanny. ``Indirect network impact on the energy consumption in multi-clouds for follow-the-renewables approaches''. In Proceedings of the 11th International Conference on Smart Cities and Green ICT Systems — SMARTGREENS 2022, pages 44–55. INSTICC, SciTePress. DOI: 10.5220/0011047000003203}. Candidate for the best paper award.

\end{itemize}

The work was also disseminated in the form of short summaries and presentations with partial results to receive initial feedback at the following events:

\begin{itemize}

\item \textit{\textbf{Miguel Felipe Silva Vasconcelos}, Daniel Cordeiro, Georges Da Costa, Fanny Dufossé, Jean-Marc Nicod, and Veronika Rehn-Sonigo. ``Long-term evaluation for sizing low-carbon cloud data centers''. In ComPAS 2023 Conférence francophone en informatique, Jul 2023, Annecy, France}

\item  \textit{\textbf{Miguel Felipe Silva Vasconcelos}. ``Towards low-carbon globally distributed clouds''. In GreenDays 2023, ENS Lyon}.

\item  \textit{\textbf{Miguel Felipe Silva Vasconcelos}, Daniel Cordeiro and Fanny Dufossé. ``Dimensioning of multi-clouds with follow-the-renewable approaches for environmental impact minimization''. In 23ème congrès annuel de la Société Française de Recherche Opérationnelle et d’Aide à la Décision (ROADEF 2022), INSA Lyon}.

\item  \textit{\textbf{Miguel Felipe Silva Vasconcelos}. ``Towards greener multi-clouds: scheduling with follow-the-renewables and dimensioning of solar panels and batteries''. In 5th Workshop of the InterSCity Project (INCT of the Future Internet for Smart Cities), São Paulo, 2022.}

\end{itemize}

Finally, the work presented in Chapter~\ref{chap:ccgrid-extension} will be submitted to a scientific journal---the publication was in the writing phase at the time this thesis was written.