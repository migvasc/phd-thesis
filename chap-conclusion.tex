\chapter{General Discussion}
\label{chap-conclusion}

\section{General Remarks and Future Research Directions}

In this thesis we aimed to ... 

contar a historinha dos capitulos,

no fim falar das possiveis work directions para trabalhos futuros ...


\section{Work Dissemination}

Many communications arose from the work performed during this thesis. The following first set presents the communications that are directly related to the contributions presented in this thesis:

\begin{itemize}
    \item \textbf{Danilo Carastan-Santos}, and Raphael Y. de Camargo. "\textit{Obtaining dynamic scheduling policies with simulation and machine learning}". In the International Conference for High Performance Computing, Networking, Storage and Analysis (SC), ACM Press, \textbf{2017}. Full paper at the main track of one of the major international conferences in High Performance Computing. This paper comprises the contributions presented in% Chapter~\ref{chap:sc} and was also nominated for both Best Paper and Best Student Paper awards of the conference.
    
    \item \textbf{Danilo Carastan-Santos}, Raphael Y. de Camargo, Denis Trystram, Salah Zrigui. "\textit{One can only gain by replacing EASY Backfilling: A simple scheduling policies case study.}". 19th IEEE/ACM International Symposium on Cluster, Cloud and Grid Computing (CCGrid), \textbf{2019}. Full paper at the main track of a well recognized international conference in High Performance Computing. This paper comrpises the contributions presented in% Chapter~\ref{chap:ccgrid} and also was the Best Paper Award winner of the conference at the aforementioned year.
\end{itemize}

The following second set presents the communications that arose in a satellite manner -- in the form of collaborations and/or supplementary work -- during the thesis. All of these works relate to the thesis subject in some way, either by the HPC resource management or the HPC applications aspects:

\begin{itemize}
    \item Luis~Sant'Ana, \textbf{Danilo Carastan-Santos}, Daniel Cordeiro and Raphael Y. de Camargo. "\textit{Real-Time Scheduling Policy Selection from Queue and Machine States.}". 19th IEEE/ACM International Symposium on Cluster, Cloud and Grid Computing (CCGrid), \textbf{2019}. Full paper at the main track of a well recognized international conference in High Performance Computing.
    
    \item \textbf{Danilo Carastan-Santos}, David C. Martins-Jr, Siang W. Song, Luiz C. S. Rozante, and Raphael Y. de Camargo. "\textit{A hybrid CPU-GPU-MIC algorithm for minimal hitting set enumeration.}". Concurrency and Computation: Practice and Experience, \textbf{2018}. Full paper in well recognized journal in High Performance Computing.
    
    \item Luis~Sant'Ana, \textbf{Danilo Carastan-Santos}, Daniel Cordeiro and Raphael Y. de Camargo. "\textit{Analysis of Potential Online Scheduling Improvements by Real-Time Strategy Selection.}". XIX Simpósio de Computação de Alto-Desempenho (WSCAD), \textbf{2018}. Full paper in a recognized Brazilian High Performance Computing workshop.
    
    \item \textbf{Danilo Carastan-Santos}, David C. Martins-Jr, Siang W. Song, Luiz C. S. Rozante, and Raphael Y. de Camargo. "\textit{A hybrid CPU-GPU-MIC algorithm for hitting set problem.}". XVIII Simpósio de Computação de Alto-Desempenho (WSCAD), \textbf{2017}. Full paper in a recognized Brazilian High Performance Computing workshop.
\end{itemize}

%discuss about the assumptions, distinctions to real scenarios and possible ways to close the work to reality





 
% - algorithms that can somehow ``think ahead'' on the scheduling decision may beat saf
 
%- Complex configurations are a reality, yet efficient algorithms are proposed only for simpler problems. The algorithms' assumptions and complexities often hinder deployment in a practical scenario. This is even a bigger problem for theoretical studies. 

%- ``Simple'' and ``transparent'' efficient solutions still remains a challenge, even in simple cases.

%- Being aware of the jobs and platform characteristics is a promising approach to improve scheduling, though it seems that non-transparent ``black-box'' ML models are to be more effective to extend the state-of-the-art performance.

%\section{General considerations}

%\section{Conclusions and future directions}