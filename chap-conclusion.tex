
Falar do problema geral ...

In this thesis we aimed to ...


This chapter begins by providing an overview of the main contributions made in this thesis, which can be found in Section~\ref{sec:conclusion_summary}. Then, in Section~\ref{sec:conclusion_future_research} it presents potential opportunities for future research directions that build upon the insights derived from the work presented in this thesis. Finally, Section~\ref{sec:conclusion_dissemination} presents the various ways in which our work has been disseminated, including a list of scientific publications that have emerged from the research conducted in this thesis.

\section{Summary of contributions }

\label{sec:conclusion_summary}

Contar a historia de cada capitulo ...

Falar da contribuicoes ...

Chap 3 ->

Smartgreens usar follow the renewables ...

Chap 4 ->
Ccgrid 
Chap 5 ->

Modelo q pode ser adotado pelos decision makers, escalavel, 
Contar a historinha dos capitulos,


\section{Future Research Directions}
\label{sec:conclusion_future_research}

No fim falar das possiveis work directions para trabalhos futuros ...

Live migration with geographic distributed dcs over the world ...

Hardwares especificos

Degradacao

As future research directions, the network usage by the workload and how it will compete for network resources with the live migrations needs further investigation. It is also possible to extend the proposed solutions for other virtualization techniques as containers by updating the estimation algorithm with a model for container live migration. Finally, some approaches explore turning off the network devices to save energy. This technique reduces the available network links in the cloud platform, and the network traffic will be re-routed, and it is necessary to analyze if the energy savings are more significant than the impacts caused by the network congestion.


In future work, we plan to propose a sizing process that also includes the IT part, for example, the new generation of servers with more powerful and energy-efficient hardware, and also their associated carbon footprint from the manufacturing process. Since this investment has been made for years, another perspective is to introduce uncertainty into this sizing process to obtain a more robust distributed DC platform that can provide satisfying service to clients even if the weather conditions change and the submitted workload evolves. The goal always being to remain as virtuous as possible. 


Alem de carbono ...

- usar outros recursos ...
- definir onde podem ser instalado dcs ...
- reciclar os servidores
- outras camadas alem da cloud ...

As future work, many research trajectories can be explored, for example: i) including degradation of the renewable infrastructure; ii) the monetary costs of servers for the decision of manufacturing new generations to replace older ones; iii) degradation and failure of specific hardware components of the servers, and the monetary costs and environmental impact of fixing and replacing them; and iv) flexibility in the scheduling could reduce the number of servers needed.


\section{Work Dissemination}
\label{sec:conclusion_dissemination}

The following two main publications emmerged from the work performed during this thesis --- both in the format of full paper:

\begin{itemize}

\item  \textit{\textbf{Vasconcelos, M.}, Cordeiro, D., Costa, G. D., Dufossé, F., Nicod, J.-M., and Rehn-Sonigo, V. (2023). Optimal sizing of a globally distributed low carbon cloud federation. In The 23rd IEEE/ACM International Symposium on Cluster, Cloud and Internet Computing. DOI: 10.1109/CCGrid57682.2023.00028}.
  
\item  \textit{\textbf{Vasconcelos, M. F. S.}, Cordeiro, D., and Dufossé, F. (2022). Indirect network impact on the energy consumption in multi-clouds for follow-the-renewables approaches. In Proceedings of the 11th International Conference on Smart Cities and Green ICT Systems — SMARTGREENS, pages 44–55. INSTICC, SciTePress. DOI: 10.5220/0011047000003203}. Candidate for the best paper award.

\end{itemize}

The work was also disseminated in the form of short-summary and presentations with partial results to receive initial feedback in the following events:

\begin{itemize}

\item \textit{\textbf{Vasconcelos, M.}, Cordeiro, D., Costa, G. D., Dufossé, F., Nicod, J.-M., and Rehn-Sonigo, V. (2023). Long-term evaluation for sizing low-carbon cloud data centers. In ComPAS 2023 Conférence francophone en informatique, Jul 2023, Annecy, France}.

  \item  \textit{\textbf{Vasconcelos, M. F. S.} (2023). Towards low-carbon globally distributed clouds. In GreenDays 2023, ENS Lyon}.

\item  \textit{\textbf{Vasconcelos, M. F. S.}, Cordeiro, D., and Dufossé, F. (2022). Dimensioning of multi-clouds with follow-the-renewable approaches for environmental impact minimization. In 23ème congrès annuel de la Société Française de Recherche Opérationnelle et d’Aide à la Décision, INSA Lyon}.

\item  \textit{\textbf{Vasconcelos, M. F. S.} (2022). Towards greener multi-clouds: scheduling with follow-the-renewables and dimensioning of solar panels and batteries. In 5th Workshop of the InterSCity Project (INCT of the Future Internet for Smart Cities), São Paulo.}

\end{itemize}

Finally, the work presented in Chapter~\ref{chap:ccgrid-extension} will be published in a scientifical journal --- the publicaton was at the writing phase at the time this thesis was written.


atualzei o modelo pra q considerasse q ficaria por 4 anos, dai 
