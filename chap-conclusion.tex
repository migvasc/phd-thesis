Cloud computing is an essential component for our modern digital society, given that it supports the marjority of services and applications we use. On the other hand, one cannot neglect the environental impact it presents, which originates both from the energy consumption of its operation --- higher than the electricity demand of entire countries --- and the life cycle of its infrastructure.

In this thesis, we studied strategies to reduce the environmental impact --- in terms of carbon footprint --- of both operating and sizing cloud federations with data centers geographically distributed over the world. More specifically, we explored two strategies that are carbon-responsive: ``follow-the-renewables'' and the sizing of the renewable infrastructure and IT equipment.

This chapter begins by providing an overview of the main contributions made in this thesis, as detailed in Section~\ref{sec:conclusion_summary}. Then, we discuss in Section~\ref{sec:conclusion_future_research} potential opportunities for future research directions that build upon the insights derived from the work presented in this thesis. Finally, Section~\ref{sec:conclusion_dissemination} presents the various ways in which our work has been disseminated, including a list of scientific publications that have emerged from the research conducted in this thesis.


\section{Summary of contributions }

\label{sec:conclusion_summary}

The first main contribution of this thesis is the analysis of the impact of the follow-the-renewables strategy in both energy consumption and network --- presented in Chapter~\ref{chap:smartgreens}. This approach is an interesting solution to exploit the geographic distribution of the cloud data centers over the world to reduce the high-carbon intensive electricity consumption. However, the decision to migrate the workload needs to consider the resources involved in this operation --- specially the network. We conducted computational experiments to investigate the impact of the follow-the-renewables strategies. These experiments used real world data for workload, climate conditions, models validated by the scientific comunity for energy consumption and network usage, and different adaptations of the follow-the-renewables.  The results demonstrated that the migration planing without taking into account the network --- in particular the network topology and the usage --- will result in network congestion and wastage of energy. As seen in Section~\ref{sec:wasted_resources_smartgreens}, the energy wasted could be used to power the servers of cloud data centers, and it also contributes to increase the environmental impact of the cloud operation. We proposed an algorithm for the migration planning that can reduce the non-renewable energy consumption while not impacting the network in comparison to works from the literature that schedule the workload migration without taking into account the network.


The second main contribution of this thesis our modeling to both size the renewable and IT infrastructure and operate cloud federations with the objective of reducing the carbonf footprint --- presented in Chapter~\ref{chap:ccgrid} and Chapter~\ref{chap:ccgrid-extension}. This model could be adopted by decision makers to guide their efforts to reduce the environmental impact of cloud computing platforms.

The initial model proposed in Chapter~\ref{chap:ccgrid} focused in the operation for the short-term with one year of duration and the possibility to manufacture on-site solar panels and lithium-ion batteries to respectively generate renewable power and store it to deal with the intermitency of green power. The fisrt key aspect of the proposed solution is that it solves both subproblems --- sizing and operating --- as a single problem, alowing to evalute decisions as increasing the on-site renewable infrastructure dimensions of a data center or scheduling the workload to the different other geographics locations using the follow-the-renwables approach. Another key aspect of the model is that it takes into account the specifc charactheristics of the geogpraphich locations of the DCs in terms of climate conditions --- the capacity to generate renewable power, cooling needs, and the local grid electricity supply --- that may already incorporate low-carbon intensive power sources. Lastly, the proposed solution does not neglect the fact that manufacturing the renewable infrastructure also presents a environmental impact. Through experiments using real-world data for workload, climate conditions, servers specifications, location of data centers, grid energy-mix,  we demonstrated that the hybrid solution, which combines the on-site renewable infrastructure and can use power from the local grid, is better in terms of carbon footprint than only operating the DCs using power from the local electricity grid or exclusively power from the on-site renewable infrastructure.

We then extended our modeling for the long-term in Chapter~\ref{chap:ccgrid-extension} --- which is mandatory given that data centers have operational lifetimes spanning over a decade. We started by improving the modeling to consider the carbon footprint of the entire life cycle of the renewable infrastructure --- from manufacturing to discarding/recycling --- which is a more accurate information of the environmental impact and reduces the over-sizing. We also presented an evaluation of including wind power in the on-site renewable source. Our experiments demonstrated that despite being able to further reduce the carbon footprint, it might not be adequate for on-site renewable production given its requirements in terms of land-area, the fact that it increases the uncertainty of the sizing process, and the higher monetary costs for wind power production. Therefore PV panels are the best candidates for the on-site renewable infrastructure and lithium-ion batteries associated to store power to use at night. We also showed that workload scheduling can mitigate a part of the impact of the erros of the sizing process --- caused by the uncertainty of the climate conditions. In terms of costs, we presented that adopting on-site renewable infrastructure is cheaper than operating the DCs using exclusively power from the local electricity grid. Additionally, the solution with solar panels and batteries is the best in both environmental and monetary aspects for the cloud operators. For the decision of manufacturing new servers given the workload we saw that greedy approaches can be used with good results .. harder to predict workload and servers speciifications... which can have an impact on the decision.

% TODO: text da conclusao pra parte de sizing ...

Finally, the model was thoughtfully designed to exclusively employ linear variables. This enables the model to be solved optimally in polynomial time and evaluate long-term scenarios with reasonlable time. This efficieny reduces the energy consumption and the associated carbon footprint of the sizing process itself.


\section{Future Research Directions}

\label{sec:conclusion_future_research}


As discussed in Section~\ref{sec:conclusion_smargreens}, to further assess the impact of the follow-the-renewables strategies, we also need to take into account the network usage by the workload --- that  reduces the avaialable bandwitdh for the migration process. Additionally, the model can be analyzed for the container virtualization technolog. Since containers are typically smaller in comparison to virtual machines, their migration time are faster. This can result in more tasks being migrated or allow migration to data centers located farther away.


We saw in Chapter~\ref{chap:smartgreens} that applying follow-the-renewables has signifcant potential for reducing the carbon emissions of the cloud operation. We didn't include it in our modeling for sizing and operating the globally distributed cloud data centers given the high latency between the network links that connect these data centers, which would result into a long migration time. This analysis needs to be made to evaluate the potential of migrating the workload, given that the DCs located farther away might have more availability of low-carbon intensive sources. One posibility is to have a threeshold for the migration time, or migrate only tasks with low priority as batch tasks.

Regarding our proposed approach for sizing and operating cloud federations, one important point for the long-term is the degradation of the renewable and IT infrastructure, as well as the failure of IT equipments. Other possibility is to analyze the sizing process considering the usage of other types of hardware for computations, in particular graphical processors (GPUs) that are widely used for artificial intelligence applications.

We explored the sizing decision for existing cloud federations. Our proposed solution could also be extended to evaluate the decision of placement of new data centers with the objecitve to minimizing the goal of the carbon footprint. For example, given a set of options for locations, their respective characteriscts in terms of potential for generating renewable power --- climate conditions, cooling needs, workload demand, and servers specifications and their associated carbon footprint, it would be possible to perform a comparative analyses of the carbon footprint of each location. The results could be used to guide the decision of the location of the new data center.


Finally, there are other types of environmental impacts than carbon emissions. For example, cloud computing platforms presents impacts in terms of water usage --- specially for cooling, resource extraction and depletion for manufacturing the IT equipments. In the first moment, it is necessary to model these types of environmental impacts to generate metrics that will permit to measure them. Similar to how there are metrics to assess the quality of service, it would be interesting to have metrics to evaluate the environmental impact of these service and applications as well. At a second step, it will be possible to propose solutions aiming to reducing the environmental impact. For example, these metrics could be incorporated into multi-objective workload scheduling algorithms or be included in the sizing process to minimize them.

\section{Work Dissemination}

\label{sec:conclusion_dissemination}

The following two main publications emmerged from the work performed during this thesis --- both in the format of full paper:

\begin{itemize}

\item  \textit{\textbf{Vasconcelos, M.}, Cordeiro, D., Costa, G. D., Dufossé, F., Nicod, J.-M., and Rehn-Sonigo, V. (2023). Optimal sizing of a globally distributed low carbon cloud federation. In The 23rd IEEE/ACM International Symposium on Cluster, Cloud and Internet Computing. DOI: 10.1109/CCGrid57682.2023.00028}.
  
\item  \textit{\textbf{Vasconcelos, M. F. S.}, Cordeiro, D., and Dufossé, F. (2022). Indirect network impact on the energy consumption in multi-clouds for follow-the-renewables approaches. In Proceedings of the 11th International Conference on Smart Cities and Green ICT Systems — SMARTGREENS, pages 44–55. INSTICC, SciTePress. DOI: 10.5220/0011047000003203}. Candidate for the best paper award.

\end{itemize}

The work was also disseminated in the form of short-summary and presentations with partial results to receive initial feedback in the following events:

\begin{itemize}

\item \textit{\textbf{Vasconcelos, M.}, Cordeiro, D., Costa, G. D., Dufossé, F., Nicod, J.-M., and Rehn-Sonigo, V. (2023). Long-term evaluation for sizing low-carbon cloud data centers. In ComPAS 2023 Conférence francophone en informatique, Jul 2023, Annecy, France}.

  \item  \textit{\textbf{Vasconcelos, M. F. S.} (2023). Towards low-carbon globally distributed clouds. In GreenDays 2023, ENS Lyon}.

\item  \textit{\textbf{Vasconcelos, M. F. S.}, Cordeiro, D., and Dufossé, F. (2022). Dimensioning of multi-clouds with follow-the-renewable approaches for environmental impact minimization. In 23ème congrès annuel de la Société Française de Recherche Opérationnelle et d’Aide à la Décision, INSA Lyon}.

\item  \textit{\textbf{Vasconcelos, M. F. S.} (2022). Towards greener multi-clouds: scheduling with follow-the-renewables and dimensioning of solar panels and batteries. In 5th Workshop of the InterSCity Project (INCT of the Future Internet for Smart Cities), São Paulo.}

\end{itemize}

Finally, the work presented in Chapter~\ref{chap:ccgrid-extension} will be published in a scientifical journal --- the publicaton was at the writing phase at the time this thesis was written.