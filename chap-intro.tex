
Cloud infrastructures became a critical component of society in the
last decade, from private life to big company development. The energy
efficiency of these platforms has been widely studied and improved by
academics and Cloud providers \cite{muralidhar2020energy}. This
progress, however, did not lead to a reduction of global Cloud energy
consumption. In~\cite{masanet2020recalibrating}, authors estimate the
growth of Data Centers (DCs) needs between 2010 and 2018 to a 10-fold
increase in IP traffic, a 25-fold increase in storage capacity, and a
6-fold increase of DCs workload. The impact of this explosion of
usages has thus been limited by efficiency improvement of platforms to
an energy increase of only 6\%. Projections over the following years
are, however, quite pessimistic. In~\cite{koot2021usage}, authors
consider different scenarios for the period 2016--2030, with
predictions ranging between a wavering balance and a significant
increase in electricity needs.


These predictions consider big trends in IT, but they do not embrace
unpredictable events, such as the COVID pandemic, and particularly the
lockdown periods, that overturned the global Information Technologies
(IT) usages \cite{feldmann2021implications}.


Another approach to reducing the environmental impact of cloud
computing energy needs consists of studying DCs' energy sources. Most
Cloud providers have made commitments to renewable energy usage in
recent years. According to a Greenpeace report, \cite{greenpeace2017},
many DCs were already fully supplied by renewable energy in
2017. However, they are not the majority of cases. A typical example
is the IT infrastructures of Virginia, which are named the ``Ground
zero for the Dirty Internet'', with 2\% of renewable energy power
plants against 37\% of coal. They are known to host 70\% of US
internet traffic. Green computing is still chimerical.





\begin{itemize}

  
   \item consumo de energia aumentando
   \item workload e evolucao de hardware (masanet)
   \item carbon aware
   \item not-so-urgent-computing
   \item inclusao de energias renovaveis nso dcs
   \item follow the renwables
   \item sizing
   \item intermitencia
   \item impacto ambiental de fabricar

\end{itemize}

   
\section{Contributions}

In this thesis we explore scheduling, sizing. More specifically, this thesis proposes the following broad sets of contributions:

\begin{enumerate}
    \item algorithms for network aware and impact of live migrations without this consideration;
    
    \item sizing
    
    \item long term sizing
    
\end{enumerate}

\section{Content}

The remainder of this thesis is organized as follows: ....

