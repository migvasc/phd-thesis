\begin{itemize}

    \item consumo de energia aumentando    
    \item workload e evolucao de hardware (masanet)
    \item carbon aware
    \item not-so-urgent-computing
    \item inclusao de energias renovaveis nso dcs
   \item follow the renwables
   \item sizing
   \item intermitencia
   \item impacto ambiental de fabricar


\end{itemize}
\section{Contributions}

In this thesis we explore scheduling, sizing. More specifically, this thesis proposes the following broad sets of contributions:

\begin{enumerate}
    \item algorithms for network aware and impact of live migrations without this consideration;
    
    %We propose a methodology to learn on-line parallel job scheduling heuristics. 
    
    \item sizing;
    
    \item long term sizing?
    
\end{enumerate}

\section{Content}

The remainder of this thesis is organized as follows: In Chapter~\ref{chap:background} we provide background knowledge, notably about scheduling and machine learning, to introduce the reader to the contributions of the thesis, as well as we present the closely related works. Chapter~\ref{chap:sc} we present the aforementioned procedure of learning scheduling heuristics, with its respective experimental results and discussions. In Chapter~\ref{chap:ccgrid} we present the aforementioned experimental campaign to provide insight on possible expectations and performance gains if one replaces EASY Backfilling, with its respective experimental results and discussions. At last but not least, in Chapter~\ref{chap-conclusion} we present a general discussion about the achieved contributions of this thesis, with the closing remarks and future research directions.

