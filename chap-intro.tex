Cloud computing has revolutionized the industrial and academic world of Information Technology with its ability to provide large amounts of computing resources on demand. Most of the applications and services we use every day that feature more and more users, such as social networking, e-mail, video games, internet banking, health applications, and video streaming, are supported by cloud computing platforms.

Despite all its benefits, cloud data centers (DCs) --- the facilities that hosts all the infrastructure needed for the cloud --- have a significant power demand. The International Energy Agency (IEA) estimates that the data centers  that host cloud computing platforms consumed between 240 to 320 TWh ---   approximately 1\% of the global final electricity generated in 2022~\cite{IEA_2022}. To put this energy consumption into perspective, it's enough to supply the anual energy demand of entire countries. In 2022 seven countries had an energy consumption similar to the DCs energy demand: Greece (316 TWh), Israel (304 TWh), Belarus (297 TWh), Switzerland (292 TWh), Hungary (266 TWh), Portugal (258 TWh) and Morocco (257 TWh)~\cite{owidenergy}.


Efforts from both academy and industry alike are trying to reduce this country-like energy consumption. An analysis of the global energy use by data centers performed by \citet{masanet2020recalibrating} showed that during the period between 2010 and 2018, the DC's internet protocol traffic increased more than 10-fold, it is estimated that the DC's storage capacity increased by 25-fold, and the DCs workload and compute instances increased by 6-fold. On the other hand, the energy consumption only increased by 6\% thanks to improvements in: i) virtualization, which allowed a 5-fold increase in the average number of compute instances by physical server; ii) more energy-efficient servers (a decrease with a factor of 4 regarding the power used for computation and a factor of 9 in terms of watts per terabyte); and iii) energy-efficient infrastructures, in particular for the cooling needs and power supply.

It is uncertain if the improvements in efficiency in cloud computing will mitigate the power increase by the growing demand for computational resources in the long term. In the work of~\citet{koot2021usage}, the authors developed a model for forecasting the energy consumption of data centers using the system dynamic models approach. Considering the period from 2016 to 2030, and the scenario with the end of Moore's law --- that predicts that the number of components in the integrated circuits doubles every year~\cite{Mack_2011_moorelaw}---  and the rise of industrial Internet of Things applications, the energy demand of the data centers could reach 2.3\% of the global electricity generation --- it will double considering what was reported by the IEA for the year 2022~\cite{IEA_2022}.

The electricity consumption of the DCs results in environmental impact in terms of greenhouse gas emissions (GHG). The extent of this impact depends on the type of source used to produce the electricity --- Table~\ref{tab:co2_power_sources} presents the emissions for the different types of power sources. In 2019, approximately 70\% of the internet's data traffic was processed by the DCs in the ``Data Center Alley'', a region located in Virginia (United States of America), which was only supplied by less than 2\% of renewable energy (green energy)~\cite{clicking_clean_virginia}. In order to reduce their environmental impact, major cloud providers --- such as Amazon AWS, Apple, Google, Meta (former Facebook) and Microsoft --- made commitments to integrate low-carbon intensive electricity produced by renewable sources in the operations of their data centers, and some already deployed their projects to reach the goal of minimizing the carbon footprint. In fact, Google reported that in 2022 64\% of the power used to supply its DCs on an hourly basis came from renewable sources~\cite{google_sustainability_report_2023}, with some DCs reaching up to 97\% of green energy usage.

\begin{table}[!ht]

\caption{ \ch{CO2} emissions from different power sources. Source~\cite{nrel_lifecycle_2021}.}\label{tab:co2_power_sources} \centering

\begin{tabular}{|l|r|}
  \hline
  \textbf{Generation technology} & \textbf{Emissions ($g\,\ch{CO2}e/kWh$)}   \\
  \hline  
  Coal   & 1 010\\  
  \hline
  Oil   & 840\\
  \hline
  Natural gas   & 486\\
   \hline
  Biomass   & 52 \\
  \hline
  Photovoltaic   & 43 \\
  \hline
  Hydropower & 21 \\
  \hline
  Nuclear   & 13 \\
  \hline
  Wind   & 13 \\
  \hline
\end{tabular}
\end{table}


The cloud platforms are integrating renewable power in their operations in the following ways. One possibility is to buy green energy from the local electricity grid, although it makes them dependent on the grid's energy mix, and not all locations in the world are supplied by low-carbon sources. Another possibility is to manufacture their own on-site renewable infrastructure. A critical aspect of producing electricity using renewable energy that must be considered for both possibilities,  is their intermittent nature --- their production fluctuates over time and is influenced by variables such as the time of the day, climate, season, and geographic location. For example, solar power is only produced when the sun is shining, and during winter, the solar irradiation is lower than in the summer, and for hydropower, long periods without rain will affect power production.


The adoption of energy storage devices (ESDs) in DCs to store renewable energy~\cite{wang2012_EDCS} is an approach to deal with intermittency. These devices are already used in DCs as a temporary backup for power outages. The green energy from the ESDs could be used when there is no production of green power, or the demand is greater than the production. Nevertheless, this approach also presents some challenges for deciding when to use or to recharge, given the characteristics of the ESDs such as the self-discharge rate (they lose stored energy over time when not used), aging (their maximum capacity decreases over time), and limited charge and discharge rates (the maximum power that can be extracted or recharged).

In this thesis, we study how to reduce the environmental impact of cloud computing data centers, specifically in terms of \ch{CO2} emissions. The usage of Carbon-Responsive strategies --- approaches that are aware of their carbon footprint and use this information to make decisions~\cite{schooler2021carbonaware} --- is explored to reduce the \ch{CO2} emissions in both operating and sizing the data centers ---  determining the required on-site renewable infrastructure (area of solar panels, number of wind turbines, capacity of ESDs). In terms of DC operations, we leverage the workload scheduling using ``follow-the-renewables'' approaches --- strategies that migrate and allocate the workload to the locations with higher renewable energy availability~\cite{shuja2016sustainable} --- and study their impact on both energy consumption and network congestion. Regarding sizing, we model our problem using a Linear Program formulation that considers the characteristics of each DC location in terms of cooling needs and local grid energy mix, as well as the fact that the renewable infrastructure generates an environmental impact. More details about the structure and contributions of this thesis are presented in the next section.
 
\section{Structure of the thesis and contributions}

This thesis is structured into six chapters, which include this introductory chapter. An overview of the content of each chapter and their respective contributions is described in what follows.

In Chapter~\ref{chap:background}, the reader is introduced to the concepts required to comprehend the research done in this thesis. In particular, the chapter contains a description of the general scenario of cloud computing, scheduling algorithms for cloud computing, the definition of carbon-responsive strategies --- in particular, the approaches explored in this thesis: ``follow-the-renewables'' and sizing, and how one can measure the environmental impact of cloud platforms. In addition, the related works used as inspiration or baseline for our proposed solutions are presented in this chapter as well.

Chapter~\ref{chap:smartgreens} presents an analysis of the adoption of the carbon-aware strategy ``follow-the-renewables'' and its impact on energy consumption and network congestion. The proposed solution consists of an algorithm that considers the network topology and usage for scheduling the workload migration. Through computational experiments with real-world data for the workload, cloud infrastructure, and a simulator that uses realistic models validated by the scientific community for both the energy consumption and network usage, the results show that the proposed solution outperforms the baselines considering the energy consumption, non-renewable energy consumption and network congestion. 

Chapter~\ref{chap:ccgrid} introduces another Carbon-Responsive strategy to reduce the carbon footprint of operating cloud computing platforms: sizing the renewable infrastructure. This strategy involves determining the required area of on-site solar panels and the capacity of the batteries to supply the DCs power demand. The proposed solution consists of a Linear Program formulation that addresses both the workload scheduling using follow-the-renewables and the sizing of the on-site renewable infrastructure. By modeling these two sub-problems as one problem it is possible to evaluate whether it is more effective to execute the workload in other geographic locations or to increase the area of solar panels and capacity of the batteries to achieve the goal of reducing the \ch{CO2} emissions. The model only uses linear variables, enabling it to be optimally solved in polynomial time --- essential for the scale of cloud computing platforms that have millions of servers and executes hundred of millions of computational tasks daily. Other important aspects of the modeling are that it considers the characteristic of each DC geographic location in terms of cooling needs and the energy mix of the local electricity grid (that may have a share of renewable energy sources), and the fact that manufacturing solar panels and batteries have an environmental impact. The results demonstrate that the hybrid configuration that combines both the on-site renewable infrastructure and the regular grid outperforms in terms of \ch{CO2} emissions the strategies of having a DC only supplied by energy produced from its local renewable infrastructure --- a reduction of approximately 30\% --- and only using power from the regular electrical grid --- a reduction of approximately 85\%.

In Chapter~\ref{chap:ccgrid-extension}, the model introduced in Chapter~\ref{chap:ccgrid} is extended to evaluate how to minimize the carbon emissions of the cloud data center operations considering the long-term. The following modifications and analysis were performed: the environmental impact of whole life cycle of the renewable infrastructure (from manufacturing, operation and recycling) is taken into account to make the model closer to the real-world; an analysis of the benefits and challenges of including wind turbines in the on-site renewable infrastructure; an assessment of the model's sensitivity to the inputs, in particular climate condiitons and grid emissions data; an anaylisis of how much the \ch{CO2} emissions can be reduced by adopting the flexibility of delaying a part of the workload, which could be used to mitiate the impacts of over-sizing the renewable infrastructure; a discussion of the monetary costs (in dolars) associated with on-site renewable infrastructure, as well as the trade-offs between minimizing the carbon footprint and reducing the costs for the DCs; and deciding to manufacture new servers over the years that may be more energy-efficient, considering that manufacturing them also emmits carbon. This modeling may be used to guide decision-makers in their efforts to reduce the carbon footprint of cloud data centers.

Finally, Chapter~\ref{chap-conclusion} presents a discussion of the main contributions of this thesis as well as possible future research trajectories.